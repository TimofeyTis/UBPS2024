\documentclass[10pt, russian]{standalone}


% стилевой пакет для написания диссертации в Latex согласно ГОСТ ТГУ
% автор к.ф.-м.н., доцент Семенова Дарья Владиславовна, dariasdv@gmail.com

%%%%%%%%%%%%%%%%%%%%%%%%%%%%%%%%%%%%%%%%%%%%%%%%%%%%%%%%%%%%%%%%%%%%55
% шрифты
\usepackage[T2A]{fontenc} %кодировки
\usepackage[english,russian]{babel} 
\usepackage[utf8]{inputenc} %кодировки
% \usepackage{literat} % Красивые русские шрифты. Инструкция по установке здесь http://blog.harrix.org/article/444
%математические шрифты и оформления
\usepackage{amsmath, amssymb, mathrsfs} 
\usepackage{amsfonts}
\usepackage{amstext}
\usepackage{amsthm}
\usepackage{cases}
\usepackage{mathtext} % Русские буквы в формулах
\usepackage[mathscr]{eucal}

%%%%%%%%%%%%%%%%%%%%%%%%%%%%%%%%%%%%%%%%%%%%%%%%%%%%5
% оформление текста
%%%%%%%%%%%%%%%%%%%%%%%%%%%%%%%%%%%%%%%%%%%%%%%%%%%%%%%%%%%%%5
% Пакеты для красивых таблиц в LaTeX
% почитать можно здесь http://mydebianblog.blogspot.ru/2013/01/advanced-tables-in-latex.html
\usepackage{booktabs} % Книжные таблицы  
\usepackage{array} % таблицы 
\usepackage{longtable} % многостраничные таблицы
\usepackage{tabularx} % стоит использовать для создания больших и сложных таблиц с разделением по страницам, с кучей текста в ячейках
\usepackage{multirow} % улучшенное форматирование таблиц
\usepackage{color,colortbl} % цветные таблицы
%%%%%%%%%%%%%%%%%%%%%%%%%%%%%%%%%%%%%%%%%%%%%%5555
% Графические пакеты 
%\usepackage{graphicx}
\usepackage{tikz}
\usetikzlibrary{calc}
\usetikzlibrary{er}
\usetikzlibrary{automata}
\usetikzlibrary{graphs}
\usetikzlibrary{intersections}
\usetikzlibrary{positioning}
\usetikzlibrary{arrows}
\usetikzlibrary{arrows.meta}
\usetikzlibrary{spy}
\usepackage{pgf}
\usepackage{pgfplots}

\pgfplotsset{compat=1.17}
%\usetikzlibrary{calc}
\usepackage{tikz-cd}
%\usepackage{pgfplots}
% \usepackage{pdfpages}
%%%%%%%%%%%%%%%%%%%%%%%%%%%%%%%%%%%%%%%%%%5
% заголовки
% \usepackage{topcapt}
%%%%%%%%%%%%%%%%%%%%%%%%%%%%%%%%%%%%%%5
%навигация 
\usepackage{makeidx} %предметный указатель
%гиперссылки
\usepackage[colorlinks, linkcolor=black, citecolor=black, urlcolor=black, pdftex, pdfhighlight=/O]{hyperref} 
% для печати надо linkcolor=black
\usepackage{cite}% управление библиографией
%%%%%%%%%%%%%%%%%%%%%%%%%%%%%%%%%%%%%%%%%%%555

% алгоритмы на псевдокоде

\usepackage[ruled]{algorithm}

%%%%%%%%%%%%%%%%%%%%%%%%%%%%%%%%%%%%%%%%%%%%



\renewcommand{\rmdefault}{ftm} % Times New Roman





%%%%%%%%%%%%%%%%%%%%%%%%%%%%%%%%%%%%%%%%%%%%%%%%%%%%%5555


%%%%%%%%%%%%%%%%%%%%%%%%%%%%%%%%%%%%%%%%%%%%%%%%%%%%%%%%%%%%%55


%%%%%%%%%%%%%%%%%%%%%%%%%%%%%%%%%%%%%%%%%%%%%%%%%%%%%%%%%%%%%%%%%%%%%%%%%%%%%%%55555
% маркер списков

\usepackage{pgfplots}
\usepackage{xcolor, colortbl}



\definecolor{sfugrey}{RGB}{255, 255, 255}
\definecolor{sfublack}{RGB}{178, 178, 178}
\definecolor{sfured}{RGB}{251, 180, 98}
\definecolor{sfuyllow}{RGB}{255,222,33}



\usetikzlibrary {patterns.meta}

\usetikzlibrary{backgrounds}

\tikzset{
  hatch size/.store in=\hatchsize,
  hatch angle/.store in=\hatchangle,
  hatch line width/.store in=\hatchlinewidth,
  hatch size=5pt,
  hatch angle=0pt,
  hatch line width=.5pt,
}


\begin{document}

\tikzset{
my dash/.style={thick,dash pattern=on 30pt off 15pt,line width=2.5pt
                }
         }% end of tikzset

    
\begin{tikzpicture}[font=\fontsize{24}{24}\selectfont]






    \def\vehicle{-- ++(2cm,0cm) -- ++(0cm,-1cm) -- ++(-2cm,0cm) -- cycle}

    \def\road {-- ++(12cm,0cm) -- ++(0cm,2cm) -- ++(-12cm,0cm) -- cycle}




    \draw[fill = sfublack, draw =  black]  node[below left =0.5cm and 0.25cm]{\textcolor{sfuyllow}{$\mathbf{1}$}} (0cm,-2cm)   \road;

    \draw[fill = sfugrey, line width = 1.5pt ]  (0cm,   -0.5cm) node[below left = 0cm and -2cm]  {$i+3$}  \vehicle ;
    \draw[fill = sfugrey, line width = 1.5pt ]  (3cm,   -0.5cm) node(i2)[below left = 0cm and -2cm]  {$i+2$}  \vehicle ;
    \draw[fill = sfugrey, line width = 1.5pt ]  (5.5cm, -0.5cm) node(i1)[below left = 0cm and -2cm]  {$i+1$}  \vehicle ;
    \draw[fill = sfugrey, line width = 1.5pt ]  (7.7cm, -0.5cm) node(i) [below left = 0cm and -1cm]  {$i$}    \vehicle ;
    \draw[ ->, line width =0.18em, arrows = {-Stealth[inset=0pt, angle=30:1.3em]}] ($(i)+(1.25cm, -0.1cm)$)   -- ++(1cm,0cm);

    \draw[line width = 2.5pt, draw = sfuyllow, dashed]  (3cm, -1.8cm)node(A){} -- ++(9cm,0cm) -- ++(0cm,1.6cm) -- ++(-9cm,0cm) node(B){}   -- cycle ;
   
    \draw[ ->, line width =0.3em, arrows = {-Stealth[inset=0pt, angle=30:1.6em]}] (12cm,-5cm)   -- ++(-12cm,0cm) node[below left = -1cm and -0.5cm] {$t$};

    \draw[line width = 1.5pt, dashed] ($(i.south)+(1.25cm, -0.1cm)$)  -- ++(0em, -3.5cm) node[above right = -0.7em and 0.1em] {$t_i$};
    \draw[line width = 1.5pt, dashed] ($(i1.south)+(1cm, 0cm)$) -- ++(0em, -3.5cm) node[above right = -0.7em and 0.1em] {$t_{i+1}$};
    \draw[<->] ($(i.south)+(1.25cm, -1.6cm)$) --node[above = -0.3em]{$\Delta t_i$} ($(i1.south)+(1cm, -1.5cm)$);

    \draw[line width = 1.5pt, dashed] ($(i2.south)+(1cm, 0cm)$) -- ++(0em, -3.5cm) node[above right = -0.7em and 0.1em] {$t_{i+2}$};
    \draw[<->] ($(i2.south)+(1cm, -2cm)$) --node[above = -0.3em]{$\Delta t_{i+1}$} ($(i1.south)+(1cm, -2cm)$);

%%%%%%%%%%%%%%%%%%%%%%%%%%%%%%%%%%%%%%%%%%%%%%%%%%%%%%%%%%%%%%%%%%%%%%%%%%%%%%%%%%%%%%%%%%%%%%%%%%%%%%%%%%%%%%%%%%%%%%%%%%%%%%%%%%%%%%%%
    \draw[fill = sfublack, draw =  black]  (0cm, 0.1cm) node[above left = 0.5cm and 0.25cm]{\textcolor{sfuyllow}{$\mathbf{2}$}}    \road;

    \draw[fill = sfugrey, line width = 1.5pt ]  (1cm,   1.5cm) node[below left = 0cm and -2cm]  {$j+2$}  \vehicle ;
    \draw[fill = sfugrey, line width = 1.5pt ]  (3.5cm, 1.5cm) node(i1)[below left = 0cm and -2cm]  {$j+1$}  \vehicle ;
    \draw[fill = sfugrey, line width = 1.5pt ]  (6.7cm, 1.5cm) node(i) [below left = 0cm and -1cm]  {$j$}    \vehicle ;
    
    \draw[line width = 2.5pt, draw = sfuyllow, dashed]  (3cm, 0.3cm)-- ++(9cm,0cm) -- ++(0cm,1.6cm) -- ++(-9cm,0cm) -- cycle ;
\end{tikzpicture}

\end{document}
