\documentclass[10pt, russian]{standalone}


% стилевой пакет для написания диссертации в Latex согласно ГОСТ ТГУ
% автор к.ф.-м.н., доцент Семенова Дарья Владиславовна, dariasdv@gmail.com

%%%%%%%%%%%%%%%%%%%%%%%%%%%%%%%%%%%%%%%%%%%%%%%%%%%%%%%%%%%%%%%%%%%%55
% шрифты
\usepackage[T2A]{fontenc} %кодировки
\usepackage[english,russian]{babel} 
\usepackage[utf8]{inputenc} %кодировки
% \usepackage{literat} % Красивые русские шрифты. Инструкция по установке здесь http://blog.harrix.org/article/444
%математические шрифты и оформления
\usepackage{amsmath, amssymb, mathrsfs} 
\usepackage{amsfonts}
\usepackage{amstext}
\usepackage{amsthm}
\usepackage{cases}
\usepackage{mathtext} % Русские буквы в формулах
\usepackage[mathscr]{eucal}

%%%%%%%%%%%%%%%%%%%%%%%%%%%%%%%%%%%%%%%%%%%%%%%%%%%%5
% оформление текста
%%%%%%%%%%%%%%%%%%%%%%%%%%%%%%%%%%%%%%%%%%%%%%%%%%%%%%%%%%%%%5
% Пакеты для красивых таблиц в LaTeX
% почитать можно здесь http://mydebianblog.blogspot.ru/2013/01/advanced-tables-in-latex.html
\usepackage{booktabs} % Книжные таблицы  
\usepackage{array} % таблицы 
\usepackage{longtable} % многостраничные таблицы
\usepackage{tabularx} % стоит использовать для создания больших и сложных таблиц с разделением по страницам, с кучей текста в ячейках
\usepackage{multirow} % улучшенное форматирование таблиц
\usepackage{color,colortbl} % цветные таблицы
%%%%%%%%%%%%%%%%%%%%%%%%%%%%%%%%%%%%%%%%%%%%%%5555
% Графические пакеты 
%\usepackage{graphicx}
\usepackage{tikz}
\usetikzlibrary{calc}
\usetikzlibrary{er}
\usetikzlibrary{automata}
\usetikzlibrary{graphs}
\usetikzlibrary{intersections}
\usetikzlibrary{positioning}
\usetikzlibrary{arrows}
\usetikzlibrary{arrows.meta}
\usetikzlibrary{spy}
\usepackage{pgf}
\usepackage{pgfplots}

\pgfplotsset{compat=1.17}
%\usetikzlibrary{calc}
\usepackage{tikz-cd}
%\usepackage{pgfplots}
% \usepackage{pdfpages}
%%%%%%%%%%%%%%%%%%%%%%%%%%%%%%%%%%%%%%%%%%5
% заголовки
% \usepackage{topcapt}
%%%%%%%%%%%%%%%%%%%%%%%%%%%%%%%%%%%%%%5
%навигация 
\usepackage{makeidx} %предметный указатель
%гиперссылки
\usepackage[colorlinks, linkcolor=black, citecolor=black, urlcolor=black, pdftex, pdfhighlight=/O]{hyperref} 
% для печати надо linkcolor=black
\usepackage{cite}% управление библиографией
%%%%%%%%%%%%%%%%%%%%%%%%%%%%%%%%%%%%%%%%%%%555

% алгоритмы на псевдокоде

\usepackage[ruled]{algorithm}

%%%%%%%%%%%%%%%%%%%%%%%%%%%%%%%%%%%%%%%%%%%%



\renewcommand{\rmdefault}{ftm} % Times New Roman





%%%%%%%%%%%%%%%%%%%%%%%%%%%%%%%%%%%%%%%%%%%%%%%%%%%%%5555


%%%%%%%%%%%%%%%%%%%%%%%%%%%%%%%%%%%%%%%%%%%%%%%%%%%%%%%%%%%%%55


%%%%%%%%%%%%%%%%%%%%%%%%%%%%%%%%%%%%%%%%%%%%%%%%%%%%%%%%%%%%%%%%%%%%%%%%%%%%%%%55555
% маркер списков

\usepackage{pgfplots}
\usepackage{xcolor, colortbl}



\definecolor{sfugrey}{RGB}{255, 255, 255}
\definecolor{sfublack}{RGB}{178, 178, 178}
\definecolor{sfured}{RGB}{251, 180, 98}



\usetikzlibrary {patterns.meta}

\usetikzlibrary{backgrounds}

\tikzset{
  hatch size/.store in=\hatchsize,
  hatch angle/.store in=\hatchangle,
  hatch line width/.store in=\hatchlinewidth,
  hatch size=5pt,
  hatch angle=0pt,
  hatch line width=.5pt,
}


\begin{document}

\tikzset{
my dash/.style={thick,dash pattern=on 30pt off 15pt,line width=2.5pt
                }
         }% end of tikzset

    
\begin{tikzpicture}[font=\fontsize{24}{24}\selectfont]






    \def\vehicle{-- ++(2cm,0cm) -- ++(0cm,1cm) -- ++(-2cm,0cm) -- cycle}

    \def\road {-- ++(12cm,0cm) -- ++(0cm,2cm) -- ++(-12cm,0cm) -- cycle}




    \draw[fill = sfublack, draw =  black] node[below left =0.5cm and 0.25cm]{$R$} (0cm,-2cm)   \road;
    \begin{scope}[shift={(0,2.1)}]
        \draw[fill = sfublack, draw =  black] node[below left =0.5cm and 0.25cm]{$L$}  (0cm,-2cm)  \road;
    \end{scope}



    \draw[fill = sfugrey] (0cm,-1.5cm)  \vehicle;
    \draw  (1cm,-1cm) node[text =  black]{$i+1$}; 

    \begin{scope}[shift={(3cm,-.0cm)},rotate=10]
        \draw[fill = sfugrey, draw = red!80, line width = 1.5pt ] (0cm,-1.5cm) \vehicle ;
        \draw[->]  (1cm,-1cm) --node[above right]{} (1.35cm,0.5cm);
        \draw[->]  (1cm,-1cm) node[left, text =  black]{$i$}--node[below right]{$a_{i}$} (5cm,-1.3cm); 
    \end{scope}
    \begin{scope}[shift={(6.5cm,2cm)}]
        \draw[fill = sfugrey, draw = red!80, line width = 1.5pt, dashed ] (0cm,-1.5cm) \vehicle ;
        \draw[->] (1cm,-1.cm)  node[left, text =  black]{$\tilde{i}$} --node[below right]{$\widetilde{a_i}$} (3cm,-1.cm);
    \end{scope}
    
    \draw[black,->, dashed, line width = 1pt] (5.2cm,-0.7cm) to [bend right=25] (7.7cm,0.5cm);
        
    \begin{scope}[shift={(9cm,-.0cm)},rotate=0]
        \draw[fill = sfugrey, text =  black] (0cm,-1.5cm) \vehicle ;
    \end{scope}

    
    \begin{scope}[shift={(0cm,2cm)}]
        \draw[fill = sfugrey, text =  black] (0cm,-1.5cm) \vehicle ;
        \draw[->] (1cm,-1.cm)  node[left, text =  black]{$j$} --node[below right]{$a_{j}$} (3cm,-1.cm);

    \end{scope}


\end{tikzpicture}

\end{document}
