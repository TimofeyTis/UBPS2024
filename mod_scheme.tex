\documentclass[11pt, russian]{standalone}



% стилевой пакет для написания диссертации в Latex согласно ГОСТ ТГУ
% автор к.ф.-м.н., доцент Семенова Дарья Владиславовна, dariasdv@gmail.com

%%%%%%%%%%%%%%%%%%%%%%%%%%%%%%%%%%%%%%%%%%%%%%%%%%%%%%%%%%%%%%%%%%%%55
% шрифты
\usepackage[T1,T2A]{fontenc} %кодировки
\usepackage[english,russian]{babel} 
\usepackage[utf8]{inputenc} %кодировки
\usepackage{pscyr} % !!! набор русских шрифтов А. Лебедева (http://tex.imm.uran.ru/texserver/fonts/pscyr/pscyr4c/). Нужен для отображения заголовков в BOLD ARIAL, BOLD ITALIC ARIAL. Этот пакет необходимо установить, если хотите смотреть Сборник УБС "as it is".

% \usepackage{literat} % Красивые русские шрифты. Инструкция по установке здесь http://blog.harrix.org/article/444
\usepackage{textcomp} % стилевой пакет textcomp, открывающий доступ к большому числу типографских значков
%математические шрифты и оформления
\usepackage{amsmath, amssymb, mathrsfs} 
\usepackage{amsfonts}
\usepackage{amstext}
\usepackage{amsthm}
\usepackage{cases}
\usepackage{mathtext} % Русские буквы в формулах
\usepackage[mathscr]{eucal}

%%%%%%%%%%%%%%%%%%%%%%%%%%%%%%%%%%%%%%%%%%%%%%%%%%%%5
% оформление текста
%%%%%%%%%%%%%%%%%%%%%%%%%%%%%%%%%%%%%%%%%%%%%%%%%%%%%%%%%%%%%5
% Пакеты для красивых таблиц в LaTeX
% почитать можно здесь http://mydebianblog.blogspot.ru/2013/01/advanced-tables-in-latex.html
\usepackage{booktabs} % Книжные таблицы  
\usepackage{array} % таблицы 
\usepackage{longtable} % многостраничные таблицы
\usepackage{tabularx} % стоит использовать для создания больших и сложных таблиц с разделением по страницам, с кучей текста в ячейках
\usepackage{multirow} % улучшенное форматирование таблиц
\usepackage{color,colortbl} % цветные таблицы
%%%%%%%%%%%%%%%%%%%%%%%%%%%%%%%%%%%%%%%%%%%%%%5555
% Графические пакеты 
%\usepackage{graphicx}
\usepackage{tikz}
\usetikzlibrary{calc}
\usetikzlibrary{er}
\usetikzlibrary{automata}
\usetikzlibrary{graphs}
\usetikzlibrary{intersections}
\usetikzlibrary{positioning}
\usetikzlibrary{arrows}
\usetikzlibrary{arrows.meta}
\usetikzlibrary{spy}
\usepackage{pgf}
\usepackage{pgfplots}

\pgfplotsset{compat=1.17}
%\usetikzlibrary{calc}
\usepackage{tikz-cd}
\usetikzlibrary{decorations.pathmorphing}
%\usepackage{pgfplots}
% \usepackage{pdfpages}
%%%%%%%%%%%%%%%%%%%%%%%%%%%%%%%%%%%%%%%%%%5
% заголовки
% \usepackage{topcapt}
\usepackage{titlesec}
\usepackage{caption} % продвинутые заголовки 
\usepackage{subcaption}
\usepackage{tocloft} % оформление оглавлений 
\usepackage{float} %плавающие объекты
%%%%%%%%%%%%%%%%%%%%%%%%%%%%%%%%%%%%%%5
%навигация 
\usepackage{makeidx} %предметный указатель
%гиперссылки
\usepackage[colorlinks, linkcolor=black, citecolor=black, urlcolor=black, pdftex, pdfhighlight=/O]{hyperref} 
% для печати надо linkcolor=black
\usepackage{cite}% управление библиографией
%%%%%%%%%%%%%%%%%%%%%%%%%%%%%%%%%%%%%%%%%%%555

% алгоритмы на псевдокоде

\usepackage[ruled]{algorithm}
\usepackage{ifthen} %простейшие средства программирования 
\usepackage{algpseudocode}
\usepackage{enumitem}
\usepackage{textcase} 


\renewcommand{\rmdefault}{ftm} % Times New Roman







    
    
      
\usepackage{diagbox}

\usepackage{pgfplots}
\usepackage{xcolor, colortbl}



\definecolor{sfugrey}{RGB}{253, 252, 247}
\definecolor{sfublack}{RGB}{4, 89, 163}
\definecolor{sfured}{RGB}{251, 180, 98}



\usetikzlibrary {patterns.meta}
\pgfdeclarepattern{
  name=hatch,
  parameters={\hatchsize,\hatchangle,\hatchlinewidth},
  bottom left={\pgfpoint{-.1pt}{-.1pt}},
  top right={\pgfpoint{\hatchsize+.1pt}{\hatchsize+.1pt}},
  tile size={\pgfpoint{\hatchsize}{\hatchsize}},
  tile transformation={\pgftransformrotate{\hatchangle}},
  code={
    \pgfsetlinewidth{\hatchlinewidth}
    \pgfpathmoveto{\pgfpoint{-.1pt}{-.1pt}}
    \pgfpathlineto{\pgfpoint{\hatchsize+.1pt}{\hatchsize+.1pt}}
    \pgfpathmoveto{\pgfpoint{-.1pt}{\hatchsize+.1pt}}
    \pgfpathlineto{\pgfpoint{\hatchsize+.1pt}{-.1pt}}
    \pgfusepath{stroke}
  }
}
\usetikzlibrary{backgrounds}

\tikzset{
  hatch size/.store in=\hatchsize,
  hatch angle/.store in=\hatchangle,
  hatch line width/.store in=\hatchlinewidth,
  hatch size=5pt,
  hatch angle=0pt,
  hatch line width=.5pt,
}



%\renewcommand\thesubfigure{(\Asbuk)}

    
      
\usepackage{diagbox}

\usepackage{pgfplots}
\usepackage{xcolor, colortbl}



\definecolor{sfugrey}{RGB}{253, 252, 247}
\definecolor{sfublack}{RGB}{4, 89, 163}
\definecolor{sfured}{RGB}{251, 180, 98}




\usetikzlibrary{backgrounds}
\usetikzlibrary{shapes, arrows, chains}

\tikzset{
  hatch size/.store in=\hatchsize,
  hatch angle/.store in=\hatchangle,
  hatch line width/.store in=\hatchlinewidth,
  hatch size=5pt,
  hatch angle=0pt,
  hatch line width=.5pt,
}


\tikzset{
	line/.style={draw, -latex'},
	every join/.style={line},
	u/.style={anchor=south},
	r/.style={anchor=west},
	fxd/.style={text width = 6em},
	it/.style={font={\small\itshape}},
	bf/.style={font={\small\bfseries}}
}
\tikzstyle{base} =
	[
		draw,
		on chain,
		on grid,
		align=center,
		minimum height=4ex,
		minimum width = 30ex,
		node distance = 6mm and 60mm,
		text badly centered
	]
\tikzstyle{coord} =
	[
		coordinate,
		on chain,
		on grid
	]
\tikzstyle{cloud} =
	[
		base,
		ellipse,
		fill = red!5,
		node distance = 3cm,
		minimum height = 2em
	]
\tikzstyle{decision} =
	[
		base,
		diamond,
		aspect=2,
		fill = green!10,
		node distance = 2cm,
		inner sep = 0pt
	]
\tikzstyle{block} =
	[
		rectangle,
		base,
		fill = blue!0,
		minimum height = 2em,
		minimum width = 5em
	]
\tikzstyle{print_block} =
	[
		base,
		tape,
		tape bend top=none,
		fill = yellow!10
	]
\tikzstyle{io} =
	[
		base,
		trapezium,
		trapezium left angle = 70,
		trapezium right angle = 110,
		fill = blue!5
	]

\tikzstyle{subroutine} =
	[
		base,
		subrtshape,
		fill = green!25
	]
\tikzstyle{cyclebegin} =
	[
		base,
		cyclebegshape,
		fill = blue!0
	]
\tikzstyle{cycleend} =
	[
		base,
		cycleendshape,
		fill = blue!0
	]
\tikzstyle{connector} =
	[
		base,
		circle,
		fill = red!25
	]


\begin{document}

\begin{tikzpicture}[%
		start chain=going below,    % General flow is top-to-bottom
		node distance=6mm and 60mm, % Global setup of box spacing
	]

    \node[block, minimum width = 7em](VD) at (0em,0em) {VEHICLE\\DETECTOR};

    \node[rectangle,draw = red, right = 2em of VD.east, dashed, minimum width = 23em, minimum height = 12em](modules)  {};
    
    \node[block, right = 0.5em of modules.west](params)  {$r(s,a)$\\MCOUNT\\TIME\_SUMM};

    \node[rectangle, right = 2em of params.east,dashed, minimum width = 6em, minimum height = 10em,draw = lightgray](block1)  {};
    \node[below = 0.5em of block1.north](block1name)  {QLEARNING};
    \node[block, below = 0.5em of block1name.south](params11)  {$p(s'\mid s,a)$\\$Q(s,a)$};
    \node[block, below = 5em of block1name.south](params12)  {$a$};


    \node[rectangle, right = 1em of block1.east,dashed, minimum width = 6em, minimum height = 10em,draw = lightgray](block2)  {};
    \node[below = 0.5em of block2.north](block2name)  {FLCONTROL};
    \node[block, below = 0.5em of block2name.south](params21)  {day\_time\\vehicles\\reward};
    \node[block, below = 5em of block2name.south](params22)  {phase};


    \node[block, below = 9em of VD.south, minimum width = 7em](SIM)  {SIMULATION};
    \node[block, below = 6.5em of params.south](TLs)  {TrafficLight1,\\TrafficLight2,\\\dots\\TrafficLightK};

    \node[above = 0.5em of modules.north]()  {\parbox{22em}{\centering\textbf{ МОДУЛЬ АДАПТИВНОГО УПРАВЛЕНИЯ\\СВЕТОФОРНЫМИ ОБЪЕКТАМИ}}};

    \path [line,line width =0.2em, arrows = {-Stealth[inset=0pt, angle=30:10pt]}] (VD.east) --node[pos=0.4, above]{\parbox{5em}{\centering tcf}}  (params);
    \path [line,line width =0.2em, arrows = {-Stealth[inset=0pt, angle=30:10pt]}] (params.east) --  ($(params.east) - (-2em,0em)$);
    \path [line,line width =0.2em, arrows = {-Stealth[inset=0pt, angle=30:10pt]}] (block1.south) |-node[pos=0.3, rotate = 90,above]{\parbox{5em}{\centering след. фаза}}   (TLs.east);
    \path [line,line width =0.2em, arrows = {-Stealth[inset=0pt, angle=30:10pt]}] (block2.south) |-node[pos=0.3, rotate = 90,above]{\parbox{5em}{\centering длинна\\след. фазы}}  (TLs.east);
    \path [line,line width =0.2em, arrows = {-Stealth[inset=0pt, angle=30:10pt]}] (TLs.west) --node[pos=0.5,above]{$s,a$}  ($(TLs.west) -  (2.6em,0em)$);
    \path [line,line width =0.2em, arrows = {-Stealth[inset=0pt, angle=30:10pt]}] ($(SIM.north) -  (-0em,0em)$) --node[pos=0.5,rotate = 90, above]{\parbox{10em}{\centering моделируемые т.с.}}  ($(VD.south) -  (-0em,0em)$);

	% \node [color = red!70,](start0) at (0em,0em)  {\textbf{МОДЕЛИРОВАНИЕ ТРАНСПОРТНЫХ ПОТОКОВ}};
	% \node [draw = red!70,cloud, below = 2.5em of start0] (start)  {Начало};
	% \node [draw = red!70,block, below = 4.5em of start, join] (phase1) {импорт дорожной\\сети map.osm};
	% \node [draw = red!70,block, below = 3.5em of phase1,  join] (phase2) {чтение\\VehicleConfig.xml};\
	% \node [draw = red!70,block, below = 3.5em of phase1, right  = 35ex of phase2 ] (phase3) {построение \\дорожного графа\\ $G=\left(N;\; E\right)$};
	% \path [line ] (phase1.east) -|  (phase3.north);

	% \node [draw = red!70,block, below = 4em of  phase3,  join] (phase4) {построение путей \\на графе};
	% \node [draw = red!70,block, below = 5em of  phase4,  join] (phase5) {расчет матриц \\корреспонденций потоков\\дорожного трафика\\ на графе};
	% \node [below  =  1em of phase5,  join] (pseudophase6) { };
	% \path [line ] (phase5) --  (pseudophase6.north);

	% \node [block,  below right  = 4.5em and 18ex  of phase5] (phase6) {\parbox{65ex}{\centering перенос параметров в имитационную модель}};

	% \node [draw = red!70,block, below = 8em of phase2] (phase7) {перенос параметров\\в модель IDM, MOBIL};
	% \path [line ] (phase2.south) -|  (phase7.north);
	% \path [line ] (phase4.west) -|  (phase7.north);

	% \path [line ] (phase7) |-  (phase6);

% 	% \draw[-, black ] ($(phase5.east)+(1em,-2em)$) --  ($(phase5.east)+(1em,18em)$);
% %%%%%%%%%%%%%%%%%%%%%%%%%%%%%%%%%%%%%%%%%%%%%%%%%%%%%%%%%%%%%%%%%%%%%%%%%%%%%%%
% 	\node [color = blue!70,right = 15em of  start0 ](start1)  {\textbf{УПРАВЛЕНИЕ СЕТЬЮ СВЕТОФОРНЫХ ОБЪЕКТОВ}};
% 	\node [draw = blue!70,block, below right  = 4.5em and 108ex of start] (phase11) {чтение\\TrafficLightConfig.xml};
% 	\path [line ] (start.south) |- ($(start.south) + (108ex,-1em)$) -- (phase11.north);

% 	\node [draw = blue!70,block, below right  = 0em and 35ex  of phase3] (phase12) {определение зон\\детекции};
% 	\path [line ] (phase3.east) --  (phase12.west);

% 	\node [draw = blue!70,block, below right  = 3.5em and 0em  of phase11] (phase13) {определение множества\\светофоров $TL$};
% 	\path [line ] (phase11.south) -|  (phase13.north);
% 	\node [draw = blue!70,block, below right  = 3.5em and 0em  of phase13,  join] (phase14) {определение функций\\обучения $Q$};
% 	\path [line ] (phase12.east) |- ($(phase12.east) + (1.5em,0)$) |-  (phase14.west);

% 	\node [draw = blue!70,block, below right  = 4em and 0em  of phase12 ] (phase15) {блок выбора\\стратегий};
% 	\node [draw = blue!70,block, below right  = 3.8em and 0em  of phase15, join ] (phase16) {совокупное \\управление $TL$};
% 	\path [line ] (phase14.south) |-   (phase16.east);

% 	\node [below   = 3.3em of phase16 ] (pseudophase16) {};
% 	\path [line ] (phase16.south) --   (pseudophase16.north);

% 	% \draw[-, black ] ($(phase14.east)+(1em,-5.5em)$) --  ($(phase14.east)+(1em,14.5em)$);
% %%%%%%%%%%%%%%%%%%%%%%%%%%%%%%%%%%%%%%%%%%%%%%%%%%%%%%%%%%%%%%%%%%%%%%%%%%%%%%

% 	\node[color = green!70,below = 36em of  start0] (start2)  {\textbf{ВЫЧИСЛИТЕЛЬНЫЕ ЭКСПЕРИМЕНТЫ}};

% 	\node [draw = green!70,block, above right  = 2.5em and 0em  of start2] (phase21) {чтение\\TrafficFlowData.csv};
% 	\path [line ] (start.south) |- ($(start.south) + (-8em,-1em)$)  |-  (phase21.west);
% 	\node [draw = green!70,block, above right  = 4em and 0em  of phase21, join] (phase22) {предобработка данных о\\ плотностях распределений\\интенсивностей  т.с. };
% 	\node [draw = green!70,block, above right  = 4.5em and 0em  of phase22] (phase23) {имитационное\\ моделирование\\ интенсивностей  т.с. };
% 	\path [line ] (phase22.north) -| (phase23.south);

% 	\path [line ] (phase23.east) -| ($(pseudophase6.south) + (0em,-2em)$) ;

% 	% \draw[-, black ] ($(phase5.east)+(1em,-6em)$) --  ($(phase5.east)+(1em,-21em)$);

% %%%%%%%%%%%%%%%%%%%%%%%%%%%%%%%%%%%%%%%%%%%%%%%%%%%%%%%%%%%%%%%%%%%%%%%%%%%%%%

% \node [color = brown!70,below = 36em of  start1] (start3)  {\textbf{ВИЗУАЛИЗАЦИЯ}};

% \node [draw = brown!70,draw = brown!70,block, below right= 12em and 0em  of phase14] (phase31) {шаг имитационной модели\\ sim.run($dt$)};
% \node [draw = brown!70,block, below right= 3.5em and 0em  of phase31] (phase34) {отрисовка кадра};
% \node [draw = brown!70,block, below right= 3em and 0em  of phase34, join] (phase37) {$t=t+dt$};
% \node [draw = brown!70,cycleend, below right  = 3em and 0em  of phase37, join] (phase32) {};
% \node [draw = brown!70,block, below left= 0em and 38ex  of phase34] (phase36) {окно pygame};


% \node [draw = brown!70,cyclebegin, above right  = 3.5em and 0em  of phase31, ] (phase30) {$t<T$};



% \path [line ] (phase30.south) --   (phase31.north);
% \path [line ] (phase31.south) --   (phase34.north);
% \path [line ] (phase34.west) --   (phase36.east);
% \path [line ] (phase6.east) -|  ($(phase30.north) - (8em,-1em)$) -|   (phase30.north);







\end{tikzpicture}
\end{document}
