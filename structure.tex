\documentclass[11pt, russian]{standalone}



% стилевой пакет для написания диссертации в Latex согласно ГОСТ ТГУ
% автор к.ф.-м.н., доцент Семенова Дарья Владиславовна, dariasdv@gmail.com

%%%%%%%%%%%%%%%%%%%%%%%%%%%%%%%%%%%%%%%%%%%%%%%%%%%%%%%%%%%%%%%%%%%%55
% шрифты
\usepackage[T1,T2A]{fontenc} %кодировки
\usepackage[english,russian]{babel} 
\usepackage[utf8]{inputenc} %кодировки
\RequirePackage{pscyr} % !!! набор русских шрифтов А. Лебедева (http://tex.imm.uran.ru/texserver/fonts/pscyr/pscyr4c/). Нужен для отображения заголовков в BOLD ARIAL, BOLD ITALIC ARIAL. Этот пакет необходимо установить, если хотите смотреть Сборник УБС "as it is".

% \usepackage{literat} % Красивые русские шрифты. Инструкция по установке здесь http://blog.harrix.org/article/444
\usepackage{textcomp} % стилевой пакет textcomp, открывающий доступ к большому числу типографских значков
%математические шрифты и оформления
\usepackage{amsmath, amssymb, mathrsfs} 
\usepackage{amsfonts}
\usepackage{amstext}
\usepackage{amsthm}
\usepackage{cases}
\usepackage{mathtext} % Русские буквы в формулах
\usepackage[mathscr]{eucal}

%%%%%%%%%%%%%%%%%%%%%%%%%%%%%%%%%%%%%%%%%%%%%%%%%%%%5
% оформление текста
%%%%%%%%%%%%%%%%%%%%%%%%%%%%%%%%%%%%%%%%%%%%%%%%%%%%%%%%%%%%%5
% Пакеты для красивых таблиц в LaTeX
% почитать можно здесь http://mydebianblog.blogspot.ru/2013/01/advanced-tables-in-latex.html
\usepackage{booktabs} % Книжные таблицы  
\usepackage{array} % таблицы 
\usepackage{longtable} % многостраничные таблицы
\usepackage{tabularx} % стоит использовать для создания больших и сложных таблиц с разделением по страницам, с кучей текста в ячейках
\usepackage{multirow} % улучшенное форматирование таблиц
\usepackage{color,colortbl} % цветные таблицы
%%%%%%%%%%%%%%%%%%%%%%%%%%%%%%%%%%%%%%%%%%%%%%5555
% Графические пакеты 
%\usepackage{graphicx}
\usepackage{tikz}
\usetikzlibrary{calc}
\usetikzlibrary{er}
\usetikzlibrary{automata}
\usetikzlibrary{graphs}
\usetikzlibrary{intersections}
\usetikzlibrary{positioning}
\usetikzlibrary{arrows}
\usetikzlibrary{arrows.meta}
\usetikzlibrary{spy}
\usepackage{pgf}
\usepackage{pgfplots}

\pgfplotsset{compat=1.17}
%\usetikzlibrary{calc}
\usepackage{tikz-cd}
\usetikzlibrary{decorations.pathmorphing}
%\usepackage{pgfplots}
% \usepackage{pdfpages}
%%%%%%%%%%%%%%%%%%%%%%%%%%%%%%%%%%%%%%%%%%5
% заголовки
% \usepackage{topcapt}
\usepackage{titlesec}
\usepackage{caption} % продвинутые заголовки 
\usepackage{subcaption}
\usepackage{tocloft} % оформление оглавлений 
\usepackage{float} %плавающие объекты
%%%%%%%%%%%%%%%%%%%%%%%%%%%%%%%%%%%%%%5
%навигация 
\usepackage{makeidx} %предметный указатель
%гиперссылки
\usepackage[colorlinks, linkcolor=black, citecolor=black, urlcolor=black, pdftex, pdfhighlight=/O]{hyperref} 
% для печати надо linkcolor=black
\usepackage{cite}% управление библиографией
%%%%%%%%%%%%%%%%%%%%%%%%%%%%%%%%%%%%%%%%%%%555

% алгоритмы на псевдокоде

\usepackage[ruled]{algorithm}
\usepackage{ifthen} %простейшие средства программирования 
\usepackage{algpseudocode}
\usepackage{enumitem}
\usepackage{textcase} 


\renewcommand{\rmdefault}{ftm} % Times New Roman







    
    
      
\usepackage{diagbox}

\usepackage{pgfplots}
\usepackage{xcolor, colortbl}



\definecolor{sfugrey}{RGB}{253, 252, 247}
\definecolor{sfublack}{RGB}{4, 89, 163}
\definecolor{sfured}{RGB}{251, 180, 98}



\usetikzlibrary {patterns.meta}
\pgfdeclarepattern{
  name=hatch,
  parameters={\hatchsize,\hatchangle,\hatchlinewidth},
  bottom left={\pgfpoint{-.1pt}{-.1pt}},
  top right={\pgfpoint{\hatchsize+.1pt}{\hatchsize+.1pt}},
  tile size={\pgfpoint{\hatchsize}{\hatchsize}},
  tile transformation={\pgftransformrotate{\hatchangle}},
  code={
    \pgfsetlinewidth{\hatchlinewidth}
    \pgfpathmoveto{\pgfpoint{-.1pt}{-.1pt}}
    \pgfpathlineto{\pgfpoint{\hatchsize+.1pt}{\hatchsize+.1pt}}
    \pgfpathmoveto{\pgfpoint{-.1pt}{\hatchsize+.1pt}}
    \pgfpathlineto{\pgfpoint{\hatchsize+.1pt}{-.1pt}}
    \pgfusepath{stroke}
  }
}
\usetikzlibrary{backgrounds}

\tikzset{
  hatch size/.store in=\hatchsize,
  hatch angle/.store in=\hatchangle,
  hatch line width/.store in=\hatchlinewidth,
  hatch size=5pt,
  hatch angle=0pt,
  hatch line width=.5pt,
}



%\renewcommand\thesubfigure{(\Asbuk)}

    
      
\usepackage{diagbox}

\usepackage{pgfplots}
\usepackage{xcolor, colortbl}



\definecolor{sfugrey}{RGB}{253, 252, 247}
\definecolor{sfublack}{RGB}{4, 89, 163}
\definecolor{sfured}{RGB}{251, 180, 98}




\usetikzlibrary{backgrounds}
\usetikzlibrary{shapes, arrows, chains}

\tikzset{
  hatch size/.store in=\hatchsize,
  hatch angle/.store in=\hatchangle,
  hatch line width/.store in=\hatchlinewidth,
  hatch size=5pt,
  hatch angle=0pt,
  hatch line width=.5pt,
}


\tikzset{
	line/.style={draw, -latex'},
	every join/.style={line1},
	u/.style={anchor=south},
	r/.style={anchor=west},
	fxd/.style={text width = 6em},
	it/.style={font={\small\itshape}},
	bf/.style={font={\small\bfseries}}
}
\tikzstyle{base} =
	[
		draw,
		on chain,
		on grid,
		align=center,
		minimum height=4ex,
		minimum width = 30ex,
		node distance = 6mm and 60mm,
		text badly centered
	]
\tikzstyle{line1} =
	[
		line,
		line width =0.13em, arrows = {-Stealth[inset=0pt, angle=30:0.6em]}
	]
\tikzstyle{coord} =
	[
		coordinate,
		on chain,
		on grid
	]
\tikzstyle{cloud} =
	[
		base,
		ellipse,
		fill = red!5,
		node distance = 3cm,
		minimum height = 2em
	]
\tikzstyle{decision} =
	[
		base,
		diamond,
		aspect=2,
		fill = green!10,
		node distance = 2cm,
		inner sep = 0pt
	]
\tikzstyle{block} =
	[
		rectangle,
		base,
		fill = blue!0,
		minimum height = 2em
	]
\tikzstyle{print_block} =
	[
		base,
		tape,
		tape bend top=none,
		fill = yellow!10
	]
\tikzstyle{io} =
	[
		base,
		trapezium,
		trapezium left angle = 70,
		trapezium right angle = 110,
		fill = blue!5
	]
\makeatletter
\pgfkeys{/pgf/.cd,
	subrtshape w/.initial=2mm,
	cycleshape w/.initial=2mm
}
\pgfdeclareshape{subrtshape}{
	\inheritsavedanchors[from=rectangle]
	\inheritanchorborder[from=rectangle]
	\inheritanchor[from=rectangle]{north}
	\inheritanchor[from=rectangle]{center}
	\inheritanchor[from=rectangle]{west}
	\inheritanchor[from=rectangle]{east}
	\inheritanchor[from=rectangle]{mid}
	\inheritanchor[from=rectangle]{base}
	\inheritanchor[from=rectangle]{south}
	\backgroundpath{
		\southwest \pgf@xa=\pgf@x \pgf@ya=\pgf@y
		\northeast \pgf@xb=\pgf@x \pgf@yb=\pgf@y
		\pgfmathsetlength\pgfutil@tempdima{\pgfkeysvalueof{/pgf/subrtshape w}}
		\def\ppd@offset{\pgfpoint{\pgfutil@tempdima}{0ex}}
		\def\ppd@offsetm{\pgfpoint{-\pgfutil@tempdima}{0ex}}
		\pgfpathmoveto{\pgfqpoint{\pgf@xa}{\pgf@ya}}
		\pgfpathlineto{\pgfqpoint{\pgf@xb}{\pgf@ya}}
		\pgfpathlineto{\pgfqpoint{\pgf@xb}{\pgf@yb}}
		\pgfpathlineto{\pgfqpoint{\pgf@xa}{\pgf@yb}}
		\pgfpathclose
		\pgfpathmoveto{\pgfpointadd{\pgfpoint{\pgf@xa}{\pgf@yb}}{\ppd@offsetm}}
		\pgfpathlineto{\pgfpointadd{\pgfpoint{\pgf@xa}{\pgf@ya}}{\ppd@offsetm}}
		\pgfpathlineto{\pgfpointadd{\pgfpoint{\pgf@xb}{\pgf@ya}}{\ppd@offset}}
		\pgfpathlineto{\pgfpointadd{\pgfpoint{\pgf@xb}{\pgf@yb}}{\ppd@offset}}
		\pgfpathclose
	}
}
\pgfdeclareshape{cyclebegshape}{
	\inheritsavedanchors[from=rectangle]
	\inheritanchorborder[from=rectangle]
	\inheritanchor[from=rectangle]{north}
	\inheritanchor[from=rectangle]{center}
	\inheritanchor[from=rectangle]{west}
	\inheritanchor[from=rectangle]{east}
	\inheritanchor[from=rectangle]{mid}
	\inheritanchor[from=rectangle]{base}
	\inheritanchor[from=rectangle]{south}
	\backgroundpath{
		\southwest \pgf@xa=\pgf@x \pgf@ya=\pgf@y
		\northeast \pgf@xb=\pgf@x \pgf@yb=\pgf@y
		\pgfmathsetlength\pgfutil@tempdima{\pgfkeysvalueof{/pgf/cycleshape w}}
		\pgfpathmoveto{\pgfqpoint{\pgf@xa}{\pgf@ya}}
\pgfpathlineto{\pgfpointadd{\pgfpoint{\pgf@xa}{\pgf@yb}}{\pgfpoint{0ex}{-\pgfutil@tempdima}}}
\pgfpathlineto{\pgfpointadd{\pgfpoint{\pgf@xa}{\pgf@yb}}{\pgfpoint{\pgfutil@tempdima}{0ex}}}
\pgfpathlineto{\pgfpointadd{\pgfpoint{\pgf@xb}{\pgf@yb}}{\pgfpoint{-\pgfutil@tempdima}{0ex}}}
\pgfpathlineto{\pgfpointadd{\pgfpoint{\pgf@xb}{\pgf@yb}}{\pgfpoint{0ex}{-\pgfutil@tempdima}}}
\pgfpathlineto{\pgfqpoint{\pgf@xb}{\pgf@ya}}
		\pgfpathclose
	}
}
\pgfdeclareshape{cycleendshape}{
	\inheritsavedanchors[from=rectangle]
	\inheritanchorborder[from=rectangle]
	\inheritanchor[from=rectangle]{north}
	\inheritanchor[from=rectangle]{center}
	\inheritanchor[from=rectangle]{west}
	\inheritanchor[from=rectangle]{east}
	\inheritanchor[from=rectangle]{mid}
	\inheritanchor[from=rectangle]{base}
	\inheritanchor[from=rectangle]{south}
	\backgroundpath{
		\southwest \pgf@xa=\pgf@x \pgf@ya=\pgf@y
		\northeast \pgf@xb=\pgf@x \pgf@yb=\pgf@y
		\pgfmathsetlength\pgfutil@tempdima{\pgfkeysvalueof{/pgf/cycleshape w}}
		\pgfpathmoveto{\pgfqpoint{\pgf@xb}{\pgf@yb}}
\pgfpathlineto{\pgfpointadd{\pgfpoint{\pgf@xb}{\pgf@ya}}{\pgfpoint{0ex}{\pgfutil@tempdima}}}
\pgfpathlineto{\pgfpointadd{\pgfpoint{\pgf@xb}{\pgf@ya}}{\pgfpoint{-\pgfutil@tempdima}{0ex}}}
\pgfpathlineto{\pgfpointadd{\pgfpoint{\pgf@xa}{\pgf@ya}}{\pgfpoint{\pgfutil@tempdima}{0ex}}}
\pgfpathlineto{\pgfpointadd{\pgfpoint{\pgf@xa}{\pgf@ya}}{\pgfpoint{0ex}{\pgfutil@tempdima}}}
\pgfpathlineto{\pgfqpoint{\pgf@xa}{\pgf@yb}}
		\pgfpathclose
	}
}
\makeatother

\tikzstyle{subroutine} =
	[
		base,
		subrtshape,
		fill = green!25
	]
\tikzstyle{cyclebegin} =
	[
		base,
		cyclebegshape,
		fill = blue!0
	]
\tikzstyle{cycleend} =
	[
		base,
		cycleendshape,
		fill = blue!0
	]
\tikzstyle{connector} =
	[
		base,
		circle,
		fill = red!25
	]


\begin{document}

\begin{tikzpicture}[%
		start chain=going below,    % General flow is top-to-bottom
		node distance=6mm and 60mm, % Global setup of box spacing
	]
	\node [color = red!70,](start0) at (0em,0em)  {\textbf{МОДУЛЬ СИМУЛЯЦИИ}};
	\node [draw = red!70,cloud, below left = 2.5em and 0em of start0] (start)  {Начало};
	\node [draw = red!70,block, below = 4.5em of start, join] (phase1) {импорт дорожной\\сети map.osm};
	\node [draw = red!70,block, below = 3.5em of phase1,  join] (phase2) {чтение\\VehicleConfig.xml};\
	\node [draw = red!70,block, below = 3.5em of phase1, right  = 35ex of phase2 ] (phase3) {построение \\дорожного графа\\ $G=\left(N;\; E\right)$};
	\path [line1] (phase1.east) -|  (phase3.north);

	\node [draw = red!70,block, below = 4em of  phase3,  join] (phase4) {построение путей \\на графе};
	\node [draw = red!70,block, below = 5em of  phase4,  join] (phase5) {расчет матриц \\корреспонденций потоков\\дорожного трафика};
	\node [below  =  1.5em of phase5,  join] (pseudophase6) { };
	\path [line1] (phase5) --  (pseudophase6.north);

	\node [block,  below right  = 4.5em and 18ex  of phase5] (phase6) {\parbox{65ex}{\centering перенос параметров в имитационную модель}};

	\node [draw = red!70,block, below = 8em of phase2] (phase7) {перенос параметров\\в модель IDM, MOBIL};
	\path [line1] (phase2.south) -|  (phase7.north);
	\path [line1] (phase4.west) -|  (phase7.north);

	\path [line1] (phase7) |-  (phase6);

	% \draw[-, black ] ($(phase5.east)+(1em,-2em)$) --  ($(phase5.east)+(1em,18em)$);
%%%%%%%%%%%%%%%%%%%%%%%%%%%%%%%%%%%%%%%%%%%%%%%%%%%%%%%%%%%%%%%%%%%%%%%%%%%%%%%
	\node [color = blue!70,right = 26em of  start0 ](start1)  {\parbox{20em}{\centering\textbf{МОДУЛЬ АДАПТИВНОГО УПРАВЛЕНИЯ}\\\textbf{СВЕТОФОРНЫМИ ОБЪЕКТАМИ}}};
	\node [draw = blue!70,block, below right  = 4.5em and 108ex of start] (phase11) {чтение\\TrafficLightConfig.xml};
	\path [line1] (start.south) |- ($(start.south) + (108ex,-1em)$) -- (phase11.north);

	\node [draw = blue!70,block, below right  = 0em and 35ex  of phase3] (phase12) {определение зон\\детекции};
	\path [line1] (phase3.east) --  (phase12.west);

	\node [draw = blue!70,block, below right  = 3.5em and 0em  of phase11] (phase13) {определение множества\\светофоров $TL$};
	\path [line1] (phase11.south) -|  (phase13.north);
	\node [draw = blue!70,block, below right  = 3.5em and 0em  of phase13,  join] (phase14) {определение функций\\обучения $Q$};
	\path [line1] (phase12.east) |- ($(phase12.east) + (1.5em,0)$) |-  (phase14.west);

	\node [draw = blue!70,block, below right  = 4em and 0em  of phase12 ] (phase15) {блок выбора\\стратегий};
	\node [draw = blue!70,block, below right  = 3.8em and 0em  of phase15, join ] (phase16) {совокупное \\управление $TL$};
	\path [line1] (phase14.south) |-   (phase16.east);

	\node [below   = 3.3em of phase16 ] (pseudophase16) {};
	\path [line1] (phase16.south) --   (pseudophase16.north);

	% \draw[-, black ] ($(phase14.east)+(1em,-5.5em)$) --  ($(phase14.east)+(1em,14.5em)$);
%%%%%%%%%%%%%%%%%%%%%%%%%%%%%%%%%%%%%%%%%%%%%%%%%%%%%%%%%%%%%%%%%%%%%%%%%%%%%%

	\node[color = green!70,below  = 34em  of  start] (start2)  {\textbf{МОДУЛЬ ВАЛИДАЦИИ}};

	\node [draw = green!70,block, above left  = 2em and 0em  of start2] (phase21) {чтение\\TrafficFlowData.csv};
	\path [line1] (start.south) |- ($(start.south) + (-8em,-1em)$)  |-  (phase21.west);
	\node [draw = green!70,block, above right  = 4em and 0em  of phase21, join] (phase22) {предобработка данных о\\ плотностях распределений\\интенсивностей  т.с. };
	\node [draw = green!70,block, above right  = 5em and 0em  of phase22] (phase23) {имитационное\\ моделирование\\ интенсивностей  т.с. };
	\path [line1] (phase22.north) -| (phase23.south);

	\path [line1] (phase23.east) -| ($(pseudophase6.south) + (0em,-1.2em)$) ;

	% \draw[-, black ] ($(phase5.east)+(1em,-6em)$) --  ($(phase5.east)+(1em,-21em)$);

%%%%%%%%%%%%%%%%%%%%%%%%%%%%%%%%%%%%%%%%%%%%%%%%%%%%%%%%%%%%%%%%%%%%%%%%%%%%%%

\node [color = brown!70,below = 36em of  start1] (start3)  {\textbf{МОДУЛЬ ВИЗУАЛИЗАЦИИ}};

\node [draw = brown!70,draw = brown!70,block, below right= 12em and 0em  of phase14] (phase31) {шаг имитационной модели\\ sim.run($dt$)};
\node [draw = brown!70,block, below right= 3.5em and 0em  of phase31] (phase34) {отрисовка кадра};
\node [draw = brown!70,block, below right= 3em and 0em  of phase34, join] (phase37) {$t=t+dt$};
\node [draw = brown!70,cycleend, below right  = 3em and 0em  of phase37, join] (phase32) {};
\node [draw = brown!70,block, below left= 0em and 38ex  of phase34] (phase36) {окно pygame};


\node [draw = brown!70,cyclebegin, above right  = 3.5em and 0em  of phase31, ] (phase30) {$t<T$};



\path [line1] (phase30.south) --   (phase31.north);
\path [line1] (phase31.south) --   (phase34.north);
\path [line1] (phase34.west) --   (phase36.east);
\path [line1] (phase6.east) -|  ($(phase30.north) - (8em,-1em)$) -|   (phase30.north);





% \path [line ,line width =0.1em, arrows = {-Stealth[inset=0pt, angle=30:10pt]}]



	%         \node [block, join] (phase1) {Этап 1};
	%         \node [cyclebegin, join=by red] (phase2) {Этап 2};
	%         \node [block, join=by green] (phase3) {Этап 3};
	%         \node [cycleend, join] (phase4) {Этап 4};
	%         \node [subroutine, join, subrtshape w = 3mm, fxd] (phase5) {Этап 5 1 2 3 4 5 6 7 8
	% 9 0};
	%         \node [io, join, fxd] (input) {Этап 6 - ввод данных};
	%         \node [block, join] (phase7) {Этап 7};
	%         \node [decision, join] (condition) {Условие};
	%         \node [connector] (finish) {Конец};
	%         % \node [block, left of = phase4, node distance = 3cm] (correction) {Коррекция};
	%         \node [print_block, right of = phase4, node distance = 3cm] (print) {Print};
	%         \path [line, red] (condition) -| node [u,near start] {Нет} (correction);
	%         \path [line] (correction) |- (phase1);
	%         \path [line] (phase2) -| node [r,near end] {Печать} (print);
	%         \path [line, green] (condition) to node [r] {Да}(finish);
\end{tikzpicture}
\end{document}
